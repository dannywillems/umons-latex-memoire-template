\documentclass[12pt,a4paper]{memoire-umons}

\usepackage[utf8]{inputenc}
\usepackage[T1]{fontenc}
\usepackage[francais]{babel}
\usepackage{amssymb,amsmath,amsthm}
%\usepackage{hyperref}% hyperliens dans le PDF, pas pour impression

\title{Titre}
\author{Prénom \textsc{Nom}}
\date{2009--2010}
\directeur{Nom du directeur}
%\directeurs{Directeur1\\ Directeur 2}
%\codirecteurs{}
\service{Service dans lequel vous avez fait votre mémoire}
\rapporteurs{
  Rapporteur 1\\
  Rapporteur 2}
%\discipline{math\'ematiques}

%%%%%%%%%%%%%%%%%%%%%%%%%%%%%%%%%%%%%%%%%%%%%%%%%%%%%%%%%%%%%%%%%%%%%%%%
%% Vos macros


%%%%%%%%%%%%%%%%%%%%%%%%%%%%%%%%%%%%%%%%%%%%%%%%%%%%%%%%%%%%%%%%%%%%%%%%

% Compile uniquement certains morceaux sans perdre les références
% automatiques et la table des matières des parties déjà compilées :
%\includeonly{introduction,chapitre1}

\begin{document}
% Éventuellement utiliser l'environnement « preface » pour avoir une
% numérotation des pages en chiffres romains.
\tableofcontents

%\include{introduction}
%\include{chapitre1}
%\include{chapitre2}
% etc

% Si vous utilisez (conseillé) BibTeX pour votre bibliographie :
%\bibliographystyle{acm}
%\bibliography{memoire}% si le fichier BibTeX est memoire.bib

\end{document}
%%% Local Variables: 
%%% mode: latex
%%% TeX-master: t
%%% TeX-PDF-mode: t
%%% End: 
